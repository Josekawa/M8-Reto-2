% Options for packages loaded elsewhere
\PassOptionsToPackage{unicode}{hyperref}
\PassOptionsToPackage{hyphens}{url}
\documentclass[
  ignorenonframetext,
]{beamer}
\newif\ifbibliography
\usepackage{pgfpages}
\setbeamertemplate{caption}[numbered]
\setbeamertemplate{caption label separator}{: }
\setbeamercolor{caption name}{fg=normal text.fg}
\beamertemplatenavigationsymbolsempty
% remove section numbering
\setbeamertemplate{part page}{
  \centering
  \begin{beamercolorbox}[sep=16pt,center]{part title}
    \usebeamerfont{part title}\insertpart\par
  \end{beamercolorbox}
}
\setbeamertemplate{section page}{
  \centering
  \begin{beamercolorbox}[sep=12pt,center]{section title}
    \usebeamerfont{section title}\insertsection\par
  \end{beamercolorbox}
}
\setbeamertemplate{subsection page}{
  \centering
  \begin{beamercolorbox}[sep=8pt,center]{subsection title}
    \usebeamerfont{subsection title}\insertsubsection\par
  \end{beamercolorbox}
}
% Prevent slide breaks in the middle of a paragraph
\widowpenalties 1 10000
\raggedbottom
\AtBeginPart{
  \frame{\partpage}
}
\AtBeginSection{
  \ifbibliography
  \else
    \frame{\sectionpage}
  \fi
}
\AtBeginSubsection{
  \frame{\subsectionpage}
}
\usepackage{iftex}
\ifPDFTeX
  \usepackage[T1]{fontenc}
  \usepackage[utf8]{inputenc}
  \usepackage{textcomp} % provide euro and other symbols
\else % if luatex or xetex
  \usepackage{unicode-math} % this also loads fontspec
  \defaultfontfeatures{Scale=MatchLowercase}
  \defaultfontfeatures[\rmfamily]{Ligatures=TeX,Scale=1}
\fi
\usepackage{lmodern}
\ifPDFTeX\else
  % xetex/luatex font selection
\fi
% Use upquote if available, for straight quotes in verbatim environments
\IfFileExists{upquote.sty}{\usepackage{upquote}}{}
\IfFileExists{microtype.sty}{% use microtype if available
  \usepackage[]{microtype}
  \UseMicrotypeSet[protrusion]{basicmath} % disable protrusion for tt fonts
}{}
\makeatletter
\@ifundefined{KOMAClassName}{% if non-KOMA class
  \IfFileExists{parskip.sty}{%
    \usepackage{parskip}
  }{% else
    \setlength{\parindent}{0pt}
    \setlength{\parskip}{6pt plus 2pt minus 1pt}}
}{% if KOMA class
  \KOMAoptions{parskip=half}}
\makeatother
\usepackage{graphicx}
\makeatletter
\newsavebox\pandoc@box
\newcommand*\pandocbounded[1]{% scales image to fit in text height/width
  \sbox\pandoc@box{#1}%
  \Gscale@div\@tempa{\textheight}{\dimexpr\ht\pandoc@box+\dp\pandoc@box\relax}%
  \Gscale@div\@tempb{\linewidth}{\wd\pandoc@box}%
  \ifdim\@tempb\p@<\@tempa\p@\let\@tempa\@tempb\fi% select the smaller of both
  \ifdim\@tempa\p@<\p@\scalebox{\@tempa}{\usebox\pandoc@box}%
  \else\usebox{\pandoc@box}%
  \fi%
}
% Set default figure placement to htbp
\def\fps@figure{htbp}
\makeatother
\setlength{\emergencystretch}{3em} % prevent overfull lines
\providecommand{\tightlist}{%
  \setlength{\itemsep}{0pt}\setlength{\parskip}{0pt}}
\usepackage{bookmark}
\IfFileExists{xurl.sty}{\usepackage{xurl}}{} % add URL line breaks if available
\urlstyle{same}
\hypersetup{
  pdftitle={Presentación -- Proyecto CRONOS2},
  pdfauthor={Jose Carlos Gomez Fernandez},
  hidelinks,
  pdfcreator={LaTeX via pandoc}}

\title{Presentación -- Proyecto CRONOS2}
\author{Jose Carlos Gomez Fernandez}
\date{2025-09-17}

\begin{document}
\frame{\titlepage}

\begin{frame}{Introducción}
\phantomsection\label{introducciuxf3n}
Este proyecto utiliza datos de CRONOS-2 Wave 4, una encuesta
internacional sobre actitudes sociales y políticas. El objetivo es
analizar la confianza en diferentes cuestiones sociales y explorar cómo
varía según factores sociodemográficos como edad, género y nivel
socioeconómico.

La presentación ofrece una visión descriptiva de los indicadores
principales y destaca patrones relevantes para comprender la relación
entre ciudadanía y opinión.
\end{frame}

\begin{frame}{Tabla resumen}
\phantomsection\label{tabla-resumen}
\pandocbounded{\includegraphics[keepaspectratio]{presentacion_files/figure-beamer/unnamed-chunk-1-1.pdf}}
\pandocbounded{\includegraphics[keepaspectratio]{presentacion_files/figure-beamer/unnamed-chunk-1-2.pdf}}
\pandocbounded{\includegraphics[keepaspectratio]{presentacion_files/figure-beamer/unnamed-chunk-1-3.pdf}}
\end{frame}

\begin{frame}{Interpretación}
\phantomsection\label{interpretaciuxf3n}
En nuestra muestra (6.032 casos):

La edad media es de 50,8 años (D.E. = 29,4).

La satisfacción con los ingresos presenta un promedio de 1,7 en una
escala de 1 a 10.

El índice medio de confianza institucional es de 4,5 en una escala de 1
a 5.

👉 Estos resultados muestran diferencias notables entre países, lo que
subraya la importancia de considerar los factores sociodemográficos y
económicos en el análisis de la confianza en las instituciones.
\end{frame}

\end{document}
